
\documentclass{article}
\usepackage [utf8x] {inputenc}
\usepackage [T2A] {fontenc}


\begin{document}

\textbf{§ 5. Геометрические приложения}

1. Касательная прямая и нормальная плоскость. Уравнение касательной прямой к кривой
$$
x=\varphi(t), y=\psi(t), z=\chi(t)
$$
в точке ее $M(x, y, z)$ имеет вид
$$
\frac{X-x}{\frac{d x}{d t}}=\frac{Y-y}{\frac{d y}{d t}}=\frac{Z-z}{\frac{d z}{d t}} .
$$
Уравнение нормальной плоскости в этой точке:
$$
\frac{d x}{d t}(X-x)+\frac{d y}{d t}(Y-y)+\frac{d z}{d t}(Z-z)=0 .
$$
2. Касательная плоскость и нормаль. Уравнение касательной плоскости к поверхности $z=f(x, y)$ в точке ее $M(x, y, z)$ имеет вид
$$
Z-z=\frac{d z}{d x}(X-x)+\frac{d z}{d y}(Y-y)
$$
Уравнение нормали в точке $M$ есть
$$
\frac{X-x}{\frac{d z}{d x}}=\frac{Y-y}{\frac{d z}{d y}}=\frac{Z-z}{-1} .
$$
Если уравнение поверхности задано в неявном виде $F(x, y, z)=0$, то соответственно имеем
$$
\frac{\partial F}{\partial x}(X-x)+\frac{\partial F}{\partial y}(Y-y)+\frac{\partial F}{\partial z}(Z-z)=0
$$
- уравнение касательной плоскости и
$$
\frac{X-x}{\frac{\partial F}{\partial x}}=\frac{Y-y}{\frac{\partial F}{\partial y}}=\frac{Z-z}{\frac{\partial F}{\partial z}}
$$
- уравнение нормали.
3. Огибающая кривая семейства плоских кривых. Огибающая кривая однопараметрического семейства кривых $f(x, y, \alpha)=0$ ( $\alpha$ - параметр) удовлетворяет системе уравнений:
$$
f(x, y, \alpha)=0, \quad f_\alpha^{\prime}(x, y, \alpha)=0 .
$$
4. Огибающая поверхность семейства поверхностей. Огибающая поверхность однопараметрического семейства поверхностей $F(x, y, z, \alpha)=0$ удовлетворяет системе уравнений:
$$
F(x, y, z, \alpha)=0, \quad F_\alpha^{\prime}(x, y, z, \alpha)=0 .
$$
В случае двупараметрического семейства поверхностей $\Phi(x, y, z, \alpha, \beta)=0$ огибающая поверхность удовлетворяет следующим уравнениям:
$$
\Phi(x, y, z, \alpha, \beta)=0, \Phi_\alpha^{\prime}(x, y, z, \alpha, \beta)=0, \Phi_\beta^{\prime}(x, y, z, \alpha, \beta)=0 .
$$
Написать уравнения касательных прямых и нормальных плоскостей в данных точках к следующим кривым:

\textbf{3528}. $x=a \cos \alpha \cos t, y=a \sin \alpha \cos t, z=a \sin t$; в точке $t=t_0$.

\textbf{3529}. $x=a \sin ^2 t, y=b \sin t \cos t, z=c \cos ^2 t$; в точке $t=\frac{\pi}{4}$.

\textbf{3530}. $y=x, z=x^2$; в точке $M(1,1,1)$.

\textbf{3531}. $x^2+z^2=10, y^2+z^2=10$; в точке $M(1,1,3)$.

\textbf{3532}. $x^2+y^2+z^2=6, x+y+z=0$; в точке $M(1,-2,1)$.

\textbf{3533}. На кривой

$$
x=t, y=t^2, z=t^3
$$
найти точку, касательная в которой параллельна плоскости $x+2 y+z=4$

\textbf{3534}. Доказать, что касательная к винтовой линии
$$
x=a \cos t, y=a \sin t, z=b t
$$
образует постоянный угол с осью $\mathrm{Oz}$.

\textbf{3535}. Доказать, что кривая
$$
x=a e^t \cos t, y=a e^t \sin t, z=a e^t
$$
пересекает все образующие конуса $x^2+y^2=z^2$ под одним и тем же углом.

\textbf{3536}. Доказать, что локсодрома
$$
\operatorname{tg}\left(\frac{\pi}{4}+\frac{\Psi}{2}\right)=e^{k \varphi} \quad(k=\text { const }),
$$
где $\varphi$ - долгота, $\psi$ - широта точки сферы, пересекает все меридианы сферы под постоянным углом.
3537. Найти тангенс угла, образованного касательной в точке $M_0\left(x_0, y_0\right)$ к кривой
$$
z=f(x, y), \quad \frac{x-x_0}{\cos \alpha}=\frac{y-y_0}{\sin \alpha},
$$
где $f$ - дифференцируемая функция, с плоскостью $O x y$.
3538. Найти производную функции
$$
u=\frac{x}{\sqrt{x^2+y^2+z^2}}
$$
в точке $M(1,2,-2)$ в направлении касательной в әтой точке к кривой
$$
x=t, y=2 t^2, z=-2 t^4 .
$$
\end{document}


